\begin{abstract}
    Due to pandemic on this year, social distancing has been recommended
    by the government, since this can reduce the spread of disease.
    With the help of technology, social distancing violation can be determined by using computational devices, such as mobile phones.
    To determine social distancing violation, firstly,
    humans are detected by using deep neural network.
    Then, the distance is calculated among detected humans.
    This dissertation proposes a social distancing detection on Android application with the purpose of
    performance maximisation by using all tools and techniques.
    Parallel computing and NEON instruction are used for performance enhancement.
    To detect humans by using deep neural network, OpenCV, a third-party library, is used in this project.
    OpenCV is used for network initialisation, and forwarding preprocessed input through the network.
    To understand performance maximisation,
    we provide comparisons among detection models, technologies, and programming languages.
    Comparisons and evaluation of the performance are presented.
    The results show that parallel computing and NEON instructions improve the application's performance.
    Processing 31 frames of video by using OpenCV without NEON instructions and using 8 threads took 10.624 seconds,
    while OpenCV with NEON took 2.890 seconds.
    In addition, the performance is better when OpenCV is manually built.
    Processing 31 frames with 8 threads by using official built OpenCV took 4.954 seconds,
    while manually built OpenCV took 2.890 seconds.
    Moreover, the performance is improved with speedup up to 52\%.
\end{abstract}