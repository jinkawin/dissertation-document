\begin{abstract}
    Due to pandemic on this year, social distancing has been recommended
    by the government, and this can reduce the spread of disease.
    With the help of technology, social distancing violation can be determined by using computational devices, such as mobile phone.
    To determine social distancing violation, firstly,
    humans are detected by using deep neural network.
    Then, the distance will be calculated among detected humans.
    This dissertation is developing social distancing detection on Android application with the purpose of
    performance maximisation by using all tools and techniques.
    Parallel computing and NEON instruction are used for performance enhancement.
    To detect humans by using deep neural network, OpenCV, a third-party library, is used in this project.
    OpenCV is used for network initialisation, and forwarding preprocessed input through the network.
    To understand performance maximisation,
    there are comparisons among detection models, technologies, and programming languages.
    Comparisons and evaluation of the performance are presented in this dissertation.
    The results show that parallel computing and NEON instruction improve the application's performance.
    In addition, the performance is better when OpenCV is manually built.
    Moreover, the performance is improved with speed-up up to 52 per cent.
\end{abstract}