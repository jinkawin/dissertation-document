\chapter{Introduction}\label{intro}
    \section{Motivation}

        % - Introduction
            There are pandemics over hundreds years, including flu pandemic in 1918 and the black death in 1346 ~\cite{REF1-01}, ~\cite{REF1-02}.
            There were millions people died during each pandemic.
            This year, humanity is facing the other pandemic, which is named as coronavirus or COVID-19.
            However, COVID-19 can be well-managed with scientific discovery and technology.
            For instance, social distancing has been recommended since there was Influenza A (H1N1) pandemic in 2009 ~\cite{REF1-05}.
            It is able to reduce the spread, and slow down and reduce the size of the epidemic peak ~\cite{REF1-03}, ~\cite{REF1-04}.
            Consequently, an infection curve is flattened, and the number of deaths is reduced.
            Yet, scientific researchers still work continuously on this outbreak.
            In addition, technology can be integrated with scientific theory to gain advantages.

        % - Problem statement


        -	Why I choose mobile application \\
            - Mobile become popular device which people use for everything (such as receiving news, study or business purpose) \\
            - 1. Performance \\
                - One technology that has high competition -> high improvement is mobile \\
                - Since the first smartphone, it has been evolved a lot \\
                - It increases capability of phone (CPU, GPU, and Memory) which is able to perform many tasks as desktop computer \\
            - 2. Portability \\
                - No charger is needed during being used \\
                - Mobile have all needed function (camera, computation hardware - CPU) \\
                - Move computation part from server to device \\
            - 3. Can be enhance by parallel computing \\

    \section{Objectives}
        -	Android application is able to do social distancing detection by using Deep Neural Network (DNN) \\
        -	Able to do the task in parallel \\
        -	Use camera to detect in the real-time \\

    \section{Structure}
        The content of this dissertation is structured as follows:
        1.	Chapter 2 provides a background knowlegde of deep neural network, parallel computing, and mobile technology. \\
        2.	Chapter 3 shows an overview of the system, data flow, and design  \\
        3.	Chapter 4 describes implementation \\
        4.	Chapter 5 provides results and analysis \\
        5.	Chapter 6 concludes xxx, limitation, and future works  \\
