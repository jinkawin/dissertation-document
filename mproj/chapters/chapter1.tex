\chapter{Introduction}\label{intro}
    \section{Motivation}

        % - Introduction
        Humanity has been faced several pandemics over hundreds of years, and many lives are threatened.
        For example, there was the black death in 1346, and the flu pandemic in 1918, and millions of people died during each pandemic ~\cite{REF1-02} ~\cite{REF1-01}.
        This year, humanity is confronting the other pandemic, which is named as coronavirus or COVID-19.

        However, a threat from illness or the impact of the outbreak can be reduced by using technology and scientific discovery.
        For instance, social distancing has been recommended since there were Influenza A (H1N1) outbreaks in 2009 ~\cite{REF1-05}.
        Social distancing is able to reduce the spread, and slow down and reduce the size of the epidemic peak ~\cite{REF1-03} ~\cite{REF1-04}.
        Consequently, the infection curve is flattened, and the number of deaths is reduced.
        Currently, scientific researchers are still working on this pandemic with the aim of reducing the infection rate.
        In addition, technology can be integrated with scientific theory to gain more advantages.

    % - Problem statement
        Due to competitive advantage and business competition, the ability of computable devices have been improved, which is capable of executing very complex tasks.
        Mobile phones, which are part of a highly competitive market, become powerful devices.
        People use mobile phones for many purposes, such as entertainment, education or business.
        Likewise, in application development, there are advantages to mobile devices.
        The first advantage is performance.
            As the aforementioned competition, the hardware of mobile phones has been improved over the years, including CPU, GPU, and memory.
            Thus, the capability of mobile phones is sufficient for performing heavy calculation tasks.
        The second advantage is portability.
            Most of the mobile phones provide necessary features, such as a camera, sensor, and GPS.
            In addition, mobile phones has a battery, which does not require a charger during being used.

    % Summary
        For the advantages above, a mobile application can be used as a tool to help human determine social distancing.

    \section{Aim and Objectives}
        % Aim
        The aim of this project is determining distances between people,
        and maximising the performance of the application to achieve the highest frame rate (frame per second or FPS).
        % Objectives
        To accomplish the aim, the objectives of this project are specified as follows:
        \begin{itemize}
            \setlength\itemsep{1em}
            \item Determine objects from the given image and video by using Deep Neural Networks (DNNs).
            \item Determine distances between people in real-time from a camera.
            \item Process inputs simultaneously by using parallel processing techniques.
            \item Use advanced single-instruction multiple-data (SIMD) NEON instructions to improve the calculation performance.
        \end{itemize}

    \section{Structure}
        The content of this dissertation is structured as follows: \\
        \begin{itemize}
            \item  Chapter 2 provides background knowledge of Mulithrading in Android application,
            social distancing determination, hardware specification, and existing applications. \\
            \item  Chapter 3 shows an overview of the system and describes implementation details in various aspects.  \\
            \item  Chapter 4 evaluates and compares the result of the performances of each implemented technology. \\
            \item  Chapter 5 concludes the results of the project, limitation of this project, and future works.  \\
        \end{itemize}
    \section{Demonstration}
    The demonstration video can be accessed from https://youtu.be/T-ZQpJyAhw4